% Realizado por Martin Cordischi
% Para Computación Paralela y Distribuida
% UNICEN 2013
\documentclass[12pt,a4paper,oneside,spanish]{report}
\usepackage{amstext}
%Comandos extras de matematica
\usepackage{mathtools}
% Paquete para anadir codigo
\usepackage{listings}
% Paquete para traducir a esp los textos automagicos
\usepackage[spanish]{babel}
\selectlanguage{spanish}
% Paquete para frames
\usepackage{framed}
% Paquete grafico
\usepackage{graphicx}
%pseudocodigo
\usepackage{algpseudocode}
\usepackage{algorithm}
\usepackage{makeidx}
%Fancy headers
\usepackage{fancyhdr}
%Encoding con tildes
\usepackage[utf8]{inputenc}

% Encabezados
%\title{}
%\author{}
%\date{}
%\maketitle
% Fin encabezados
%\makeindex


%Header and footer
\thispagestyle{empty}
\renewcommand{\headrulewidth}{0.0pt}
\pagestyle{fancy}
\lhead{}
\chead{}
\rhead{}
\lfoot[]{}
\chead[]{}
\rfoot[]{}

\newcommand{\HRule}{\rule{\linewidth}{0.5mm}}
\begin{document}

\begin{titlepage}
\begin{center}

% Upper part of the page. The '~' is needed because \\
% only works if a paragraph has started.
%\includegraphics[width=0.5\textwidth]{cuys.eps}~\\[1cm]

\textsc{\LARGE UNICEN}\\[1.5cm]



% Title
\HRule \\[0.4cm]
{ \huge \bfseries Copay-di}\\[0.4cm]

\HRule \\[0.4cm]

% Author and supervisor

\begin{center}
\textbf{}
\end{center}
~\\[3.5cm]


\textit{}


~\\[0.4cm]
\textit{} \textbf{} 

\vfill

% Bottom of the page
{\large \textbf{UNICEN} - Junio 2013}

\end{center}
\end{titlepage}

\section*{Resumen}


